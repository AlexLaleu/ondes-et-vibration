\documentclass[a4paper,11pt]{article}
\usepackage[utf8]{inputenc}
\usepackage{amsmath, amssymb, amsfonts}
\usepackage{geometry}
\geometry{margin=2.5cm}
\usepackage{mathtools}

\begin{document}
	
	\section*{Résolution numérique de l'équation de vibration d'une membrane}
	
	Le but de ces codes est de résoudre numériquement l’équation de vibration d’une membrane, en utilisant la méthode des éléments finis pour la discrétisation spatiale, et le schéma de Newmark pour la résolution en temps.
	\\ \\
	On s’intéresse aux vibrations d’une membrane circulaire de tension uniforme \(T\), de masse surfacique \(\sigma\), et de rayon extérieur \(R\). L’équation aux dérivées partielles régissant le déplacement transverse \(u(r, \theta, t)\) s’écrit, pour tout temps \(t\) et tout point d’espace \((r, \theta) \in \Omega\), avec \(\Omega\) le disque de rayon \(R\) :
	
	\[
	\sigma \frac{\partial^2 u}{\partial t^2} - T \Delta u = 0.
	\]
	
	On suppose qu’au bord \(r = R\), le mouvement est nul : 
	\[
	u(R, \theta, t) = 0.
	\]
	
	\subsection*{Formulation variationnelle}
	
	La formulation variationnelle de ce problème s’écrit :
	\[
	 \forall v \in H^1_0(\Omega), \; \forall t \in ]0, t_{\text{max}}[, \quad \int_{\Omega} \sigma \frac{\partial^2 u}{\partial t^2} v(x) \, d\Omega(x) 
	+ T \int_{\Omega} \nabla u(x, t) \cdot \nabla v(x) \, d\Omega(x) = 0
	\]
	
	\subsection*{Semi-discrétisation spatiale}
	
	On procède à une triangulation \(\mathcal{T}_h\) du domaine \(\Omega\), et \(V_h\), l’approximation de \(H^1(\Omega)\) par des éléments finis \(P_1\) associés à la triangulation \(\mathcal{T}_h\). 
	
	On note \((T_\ell)_{\ell=1,L}\) les triangles de \(\mathcal{T}_h\), \((M_I)_{I=1,N}\) les sommets des triangles, et \((w_I)_{I=1,N}\) la base de \(V_h\) définie par :
	\[
	w_I(M_J) = \delta_{IJ}, \quad 1 \leq I, J \leq N.
	\]
	
	On appelle \(V_h^0\) le sous-espace de \(V_h\) qui est inclus dans \(H^1_0(\Omega)\). Ce sous-espace est engendré par l’ensemble des fonctions de base associées à un nœud intérieur :
	\[
	V_h^0 = \text{Vect} \{ w_I \; | \; M_I \notin \partial \Omega \},
	\]
	et on note \(N_0\) la dimension de \(V_h^0\). \\ \\
	 \(V_h^0\) représente le maillage dans nos codes. Nous en aurons trois : $h \in \{0.1, 0.2, 0.5\}$.Nous visualiserons le mouvement pour h=0.2 mais les autres maillages nous permettons de vérifier la convergence des valeurs propres des matrices de masse et de raideur selon le raffinement du maillage.\\ \\
	La solution approchée \(u_h\) s’écrit sous la forme :
	\[
	\forall (x, y) \in \Omega , \quad u_h(x, y; t)  = \sum_{w_I \in V_h^0} u_h(M_I; t) w_I(x, y) 
	\]
	
	\subsection*{Formulation variationnelle discrète}
	
	Le problème semi-discrétisé en espace sous la forme matricielle :
	\[
	\sigma \mathbf{M} \ddot{\mathbf{u}} + T \mathbf{K} \mathbf{u} = 0,
	\]
	où :
	\[
	K_{IJ} = \int_{\Omega} \nabla w_I(x, y) \cdot \nabla w_J(x, y) \, d\Omega, \quad 1 \leq I, J \leq N_0  \quad , \quad  M_{IJ} = \int_{\Omega} w_I(x, y) w_J(x, y) \, d\Omega, \quad 1 \leq I, J \leq N_0
	\]
	- \(\mathbf{u}\) est le vecteur contenant les coefficients des solutions discrètes.\\
	- \(\mathbf{M}\) est la matrice de masse.\\
	- \(\mathbf{K}\) est la matrice de raideur.\\
	
	Dans toute la suite, on prendra \(\sigma = 1\) et \(T = 1\).
	\subsection*{Résolution temporelle}
	La résolution temporelle se fait avec le schéma de Newmark où on choisit les constantes $\gamma=0.5$, $\beta=0.25$, $dt=0.01$. Nous résoudrons entre les temps t=0 et t=10s pour la condition initiale suivante :
	\[
	y(r, \theta) = 
	\begin{cases} 
		\frac{A}{2} (1 + \cos(\frac{ \pi r}{r_0})), & \text{si } r \leq r_0, \\
		0, & \text{sinon},
	\end{cases}
	\]
	où \(A\) représente l’amplitude de la condition initiale et \(r_0\) son extension spatiale.
	
	On choisira \(A = 2\) et \(r_0 = \frac{1}{2}\).
\end{document}
