\documentclass[a4paper,11pt]{article}
\usepackage[utf8]{inputenc}
\usepackage{amsmath, amssymb, amsfonts}
\usepackage{mathtools}

\begin{document}
	
	\section*{Simulation du mouvement d'une corde verticale}
	
	On considère le cas d'une corde verticale dont les deux extrémités sont fixes. La tension subie par la corde est uniquement le fruit de la gravité. La corde possède une masse linéique $\rho$ constante. \\
	Nous allons décomposer le mouvement de notre corde verticale avec gravité sur les modes normaux d'une corde horizontale sans gravité. Ces modes normaux sont : $\phi_n(x)=\sqrt(2)sin(n\pi x)$\\
	Le déplacement s'écrit alors sous la forme : 
	\[
	y(x,t)=\sum_{n=1}^{+\infty} \phi_n(x)q_n(t)
	\]
	L'équation du mouvement sous forme adimensionnelle projetée sur les modes normaux devient :
	\[
	\ddot{q}_m + m^2\pi^2q_m + \sum_{n=1}^{+\infty} \left(a_{mn} + b_{mn}\right)q_n = 0.
	\tag{1}
	\]
	Où on a posé 
	\[
	a_{mn} = -2n^2\pi \displaystyle\int_{0}^{1} x \sin(m\pi x)\sin(n\pi x)\,dx = 
	\begin{cases}
		= -n^2 \left[\frac{(-1)^{m-n} -1}{(m-n)^2} - \frac{(-1)^{m+n} -1}{(m+n)^2}\right], & \text{si } n \neq m, \\[10pt]
		-\frac{n^2\pi^2}{2}, & \text{si } n = m.
	\end{cases}
	\]
	\[
	b_{mn}=2n\pi \int_{0}^{1} sin(m\pi x)cos(n\pi x)dx=
	\begin{cases}
		0, & \text{si } m = n, \\[10pt]
		-n \left( \frac{(-1)^{m-n} - 1}{m-n} + \frac{(-1)^{m+n} - 1}{m+n} \right), & \text{si } m \neq n.
	\end{cases}
	\]
	Dans la mesure où la base modale d’une corde avec tension constante n’a aucune raison d’être aussi la base modale de la corde avec gravité, il n’est pas étonnant que les équations soient couplées. L’objet de ces codes est de calculer la base modale du nouveau problème, soit l’ensemble des fonctions $\psi_n(x)$, ceci en fonction de la base de départ (fonctions $\phi_n(x)$).
	
	\subsection*{Troncature à \(N\) modes et équations matricielles}
	
	On décide à présent de faire une troncature à \(N\) modes de l’équation \((1)\). On peut alors écrire les équations \((1)\) sous forme matricielle dans la base $\phi$ :
	\[
	\mathbf{M}\ddot{\mathbf{q}} + \mathbf{K}\mathbf{q} = 0.
	\tag{2}
	\]
	
	La matrice $\mathbf{K}$ est pleine et les équations sont couplées. La recherche de solutions de la forme
	\[
	\mathbf{q}(t) = \mathbf{V} e^{-i\omega t},
	\]
	conduit au problème aux valeurs propres suivant :
	\[
	\left(\mathbf{K} - \omega^2 \mathbf{M}\right)\mathbf{V} = 0.
	\tag{3}
	\]
	
	Les valeurs propres et vecteurs propres de ce système linéaire donnent le carré des pulsations propres et les modes propres correspondants. Attention, ces vecteurs propres sont exprimés dans la base des fonctions $\phi_n(x)$.
	\\ \\
	Nous allons calculer l’évolution du déplacement \(y(x,t)\) au cours du temps en prenant pour condition initiale :
	\[
	y(x, 0) = \sqrt{2}\sin(\pi x), \quad \dot{y}(0, t) = 0.
	\]
	
	On remarquera ici qu’il s’agit d’une condition initiale telle que le vecteur \(\mathbf{q}\) vaut 0 partout sauf sur son premier terme, où il vaut 1. 
	
	Le calcul se fera dans la base \(\psi\), dans laquelle le problème est découplé. Il s’agira donc de calculer le vecteur \(\mathbf{r}\) correspondant en utilisant les matrices de passage. Une fois l’évolution calculée dans la base des fonctions \(\psi_n\), il s’agira de passer de la base \(\psi_n\) au déplacement de la corde aux points de discrétisation.
	
\end{document}
